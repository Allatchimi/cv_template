\documentclass[11pt,a4paper,sans]{moderncv}
% taille de police ('10pt', '11pt' et '12pt'),
% taille de la feuille ('a4paper', 'letterpaper', 'a5paper', 'legalpaper', 'executivepaper' et 'landscape')
% la famille de fonte ('sans' et 'roman')

% Choix du thème et de la couleur
\moderncvstyle{classic} % 'casual' 'classic' 'oldstyle' 'banking'.
\moderncvcolor{blue} % 'blue' (par défaut)'orange' 'green' 'red' 'purple' 'grey' 'black'.
% Largeur de la colonne date
\setlength{\hintscolumnwidth}{2cm} 

% Extensions additionnelles
\usepackage[utf8]{inputenc}
\usepackage[scale=0.8]{geometry}
\usepackage{helvet}
\usepackage[french]{babel}
\usepackage{comment}

% Données personnelles

\name{MAHAMAT}{Amine}
%\title{titre complémentaire} 
\address{12, rue de la gare}{77\,000 Provins}{Cameroun}
\phone[mobile]{+237XXXXXX}
%\phone[fixed]{01~01~88~33~55}
%\phone[fax]{02~11~22~33~44}
\email{addresse@gmail.com}
\homepage{www.site.com}
\social[linkedin]{pierre.durand}
\social[twitter]{pierre.durand}
\social[github]{https://github.com/github}
%\extrainfo{informations complémentaires.}
%\quote{Encore un titre}
% Insertion photo
\photo[64pt][0.4pt]{amine1.jpg}


% Début du document
\begin{document}
\makecvtitle

% Corps du CV
\section{Formation}
\cventry{1999--2000}{Baccalauréat Série D}{Lycée De Martyre de Faya}{Brest}{\textit{Passble}}{Option Sciences}
\cventry{2000--2005}{Informatique}{Université de yaounde 1}{Sens}{\textit{Ingénieur réseau}}{Description}

\section{Compétences Techniques}
\begin{itemize}
    \item \textbf{Langages de Programmation} : Java, Python, JavaScript
    \item \textbf{Frameworks et Outils} : Spring Boot, Spring Security, Spring Data JPA, Hibernate
    \item \textbf{Base de Données} : MySQL, PostgreSQL
    \item \textbf{Outils de Versionnage} : Git, GitHub
    \item \textbf{Autres} : REST API, JSON, Maven, Docker
\end{itemize}


\section{Expériences}
\cventry{}{Stage}{Stage : Développeur Back-End (3 mois)}{Yaounde}{IREX}%
{%
\begin{itemize}%
\item  Mise en œuvre et procédés en salle blanche;
\item intégration et caractérisation des membranes MEMS d'épaisseur nanométrique
  \begin{itemize}%
  \item AFM;
  \item Vibromètre laser ;
  \item MEB.
  \end{itemize}
\end{itemize}}
\cventry{}{Projet Académique}{Yaounde}{Université de Yaounde 1}{}{%
\begin{itemize}
\item Développement d'une application web permettant la gestion des cours et des étudiants;
\item Mise en place de la persistance des données avec Spring Data JPA et PostgreSQL;
\item Gestion des utilisateurs avec rôles différenciés (admin, professeurs, étudiants).
\item Documentation de l'API avec Swagger.
\end{itemize}}

\section{Certifications}
\begin{itemize}
    \item Spring Framework Certification (Coursera/Pluralsight)
    \item Oracle Certified Associate, Java SE 8 Programmer
\end{itemize}

\section{Langues}
\cvitemwithcomment{Français}{Lu, parlé, écrit}{un commentaire si besoin}
\cvitemwithcomment{Arabe}{Lu, parlé, écrit}{Idem}
\cvitemwithcomment{Anglais}{Lu, parlé, écrit}{Idem}



\begin{comment}
\section{Compétences Techniques}
\cvdoubleitem{Java}{blabla, blabla}{C++}{blabla, blabla}
\cvdoubleitem{Php}{blabla, blabla}{Pascal}{blabla, blabla}
\cvdoubleitem{\LaTeX}{blabla, blabla}{Python}{blabla, blabla}

\end{comment}

\section{Activités et Intérêts}
\cvitem{Hackathons}{Participation à des hackathons (thèmes : IoT, développement web).}
\cvitem{Lecture}{ Lecture d'ouvrages sur les architectures logicielles modernes.}

\end{document}ppp
